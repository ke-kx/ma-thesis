\chapter{Introduction}\label{ch:intro}

\todo{benny stolpert über ``outsource'':
Developers often use existing libraries or software frameworks, which expose their functionality through an application programming interface (API), to take care of the most complex tasks.
These APIs would ideally be simple and easy to use, but there is always a trade-off between usability and flexibility.
}
%--- software systems want to use APIs to facilitate reuse, however, using an API is not always trivial
Developers often outsource a significant amount of work to existing libraries or software frameworks, which expose their functionality through an application programming interface (API).
These APIs would ideally be simple and easy to use, but there is always a trade-off between usability and flexibility.
More powerful frameworks in particular struggle to provide a trivial API without limiting developers' ability to take greatest advantage of their functionality.
Correctly invoking an API often requires that a developer know it well;
not considering specific constraints and requirements can lead to serious bugs or complications.

%--- since it is often not trivial to use an API correctly, there will be ``wrong'' usages + examples?
Sometimes, these problems could have been avoided if the developer had paid better attention to the API's documentation.
However, documentation is often vague or incomplete leading to misinterpretations or oversights.
Moreover, even if it is of high quality, the complexities of some libraries make it inevitable that there will be erroneous invocations of their APIs.
Regardless of their origin, mistakes can relate to parameter choice, method order, or a range of other factors (some methods may need to be invoked in an extra thread, a specific precondition may have to be satisfied, setup work may need to be performed).

Examples, which come to mind in Java, are a programmer calling \code{next()} on an iterator without first checking with \code{hasNext()} if it even contains another object, or a class which overrides \code{equals()} without ensuring that an invocation of \code{hashCode()} always produces the same output for two equal objects.
%--- correct usages of an API often share some underlying patterns + examples of them
Despite the fact that there might be numerous appropriate ways to use an API, there are underlying patterns that the correct invocations have in common.
% These patterns can take a lot of different forms and shapes, for example ``call method \code{foo} before calling method \code{bar}'', ``if an object of type \code{Foobar} is used as a parameter to method \code{baz}, condition X has to be fulfilled'', ``never call method \code{qux} in the GUI thread'', etc.
For the Java examples, they could take the form of ``an object which implements \code{equals()} must also implement \code{hashCode()}'', ``if object A equals object B, their \code{hashCode()} must also be equal'' or ``a call to \code{next()} on an iterator should be preceded by a call to \code{hasNext()} to ensure that it actually contains another object''.

%--- if we can extract the patterns, hopefully, the wrong usages will stand out and enable us to detect bugs
API usage patterns~\cite{robillard2013automated} can aid in detecting these kinds of defects in a software project, by noting where the code deviates too far from the patterns associated with the API it is employing.
The challenge here is twofold.
First, it is extracting and recognizing the patterns.
Second, it is understanding when a deviation from a pattern actually signifies a code smell or an error that should be corrected.

\section{Missing Method Calls}
% Motivating the research aka which problem are we trying to solve, general description of the problem

%--- we will focus on MMCs + top-level view of missing method calls + example
Of the many subtle mistakes a developer can make when invoking an API, this work focuses on one specific type: missing method calls, which occur in the context of Object Oriented Programming (OOP).
In OOP software, an object of a specific type is almost always used by invoking some of its methods.
Types will then have underlying patterns, such as ``when methods \code{A} and \code{B} of type $T_1$ are invoked, then method \code{C} is called as well'' or ``methods \code{X} and \code{Y} of type $T_2$ are always used together''.
Given an object of type $T_1$ on which only methods \code{A} and \code{B} are called, we can say that a call to \code{C} is missing.
Similarly, if we have an object of type $T_2$ where only \code{X} is invoked, we can say that a call to \code{Y} is missing.

%-- not clear what's the point of this one
Developers often fail to make important method calls when using classes with which they are not deeply familiar.
As this can occur with a new library, an extensive framework, or even a whole platform (e.g., Android), we use these terms interchangeably.
Lack of familiarity may also arise not just when using code from an external source, but also when working on a big codebase.
In large software projects it is common that one developer does not know all of the code, and thus, they will often encounter classes which they do not know how to use correctly.

%--- Monperrus et al. also showed that the bugs sometimes DO get into the code, all in all, we want to be able to detect them
Studies have confirmed the intuition that missing method calls are common pitfalls across a range of applications and even in mature code bases.
In an informal review~\cite{monperrus2010detecting}, Monperrus et al.\ found bug reports\footnote{\url{https://bugs.eclipse.org/bugs/show_bug.cgi?id=222305}} and problems\footnote{\url{https://www.thecodingforums.com/threads/customvalidator-for-checkboxes.111943/}} related to missing method calls in many newsgroups, bug trackers, and forums.
The issues range from runtime exceptions\footnote{\url{https://issues.apache.org/jira/browse/TORQUE-42}} to problems in some limit cases, but generally reveal at least a code smell, if not worse.
They also performed an extensive analysis~\cite{monperrus2013detecting} of the Eclipse bug repository,
searching for syntactic patterns that they deemed related to missing method calls, such as: ``should call'', ``does not call'', ``is not called'', or ``should be called''.
Manual inspection confirmed that more than half (117 of 211) of the bug reports located in this manner were indeed related to a missing method call.

This number is if anything an underestimation of the total number of missing method call bugs in the code base.
After all, the researchers might have missed some syntactic patterns, and the bug repository can only contain those bugs that have already been discovered.
Given how common they are, it seems clear that automatically detecting missing method calls in production code would be very helpful, not only saving developer time but also making maintenance cheaper and more manageable.

\section{Detection -- but how?}

% General idea why we want to use Recommender Systems / Learning!
%--- the first idea for detection: static rules, but the problem: very time/cost intensive
A simple and straightforward approach to detecting missing method calls would be to build a set of hard-coded rules, such as:
\begin{itemize}
    \item ``always call \code{setControl()} after instantiating a \code{TextView}''
    \item ``in Method \code{onCreate()} of classes extending \code{AppCompatActivity} always call \code{setContentView()}''
    %\item ``when calling \code{foo()} also call \code{bar()}''
    \item ``when calling \code{next()} on an Iterator always call \code{hasNext()} (before)''
\end{itemize}
Well-crafted and -reasoned rules along these lines could facilitate precise detection of missing method calls and contribute to better, less buggy code.
However, creating and maintaining a list of rules like this is a manual effort and would require tremendous amounts of time and money, especially in a world where software is changing continually.
While the effort might even be justified for large and important libraries, it multiplies with the size of the library until becoming completely infeasible.

%11 solution: automatic detection, even if it has some drawbacks
Instead, we propose automatically detecting locations in a code base where a method call is potentially missing, using only the code itself as input.
Such an approach would adapt to changes without requiring additional work from a developer and could also be applied to proprietary code, which is closed to the public.
The locations this method pinpoints might not be as accurate as those discovered by a hand-crated list of rules, bug a second step could address this problem: manual examination by an expert, who would determine the severity of the problem and, if necessary, issue a fix.
Whether this proves to save time and money hinges on the accuracy of the approach.

\todo[inline]{clearly formulate problem statement somewhere (3-5 sentences which describe the problem to be solved + the ansatz) -> i think this is fine with the paragraph above?}

%--- additional advantages to automatic detection: continuous integration, adaptability, can be used on closed software, etc
%\todo[inline]{additional advantages to automatic detection: continuous integration, adaptability, can be used for closed software
%-> express some more much better than fixed preprogrammed rules, can adapt to changing system, be specific for own closed library, etc}

%--- how this work relates to recommender systems
% the approach was chosen in this work bases of recommender systems / learning
% (Mention the ideas of \cite{bruch2012ide}, chapter 2 as an inspiration / the way to the idea - maybe)
% first find likely recommendations, for writing, then realize, if something is super likely given a particular situation, but it is not there, it seems like a good indicator of an error

\section{Contribution}

To understand and tackle the missing method call problem, Monperrus et al.~\cite{monperrus2010detecting}\cite{monperrus2013detecting} introduced the notion of type usages.
A type usage is the list of method calls which are invoked on an object of some type and occur in the body of some method.
The idea behind the technique by Monperrus et al.\ is to check for outliers among the type usages by using the majority rule:
If a type is used in one particular way many, many times (that is, in the majority of cases) and differently only once (or a few times), this probably indicates a bug.
% This thesis builds on their work, re-evaluating the characterization of type usages and anomalies among them.

\todo[inline]{
    the part above is not actually contribution, but seems fine as intro, but add lots of meat for the next paragraph!
    needs more content! what am i actually doing and why (hoping to find what?) and what are the results?
    implementation also part of my work! -> part of contribution!!!

take their implementation and improve it
based on their method some related variants
evaluation with the goal of understanding which of the variants works best
qualitative evaluation on android
+ generally get a feeling for the performance if this method
understand if automated benchmark works + some other properties that this method entails

evaluation -> more detail + what are the QUESTIONS?
}

We evaluate this concept by applying it to a dataset of more than 600 open source Android applications and performing a manual evaluation of the results.
We further experiment with small changes to their initial idea and put them under the scrutiny of an automated benchmark.
Additionally, we compare the results of the manual evaluation against those of the automated one and consider what this means for future research.

\section{Outline}

In Chapter~\ref{ch:relWork} we present previous work which goes in a similar direction.
Chapter~\ref{ch:dmmc} explains the theoretical background of the method proposed by Monperrus et al.\ and in Chapter~\ref{ch:ext} we propose some variations to their method.
We explain the inner workings of our implementation in Chapter~\ref{ch:impl}, before summarizing the details and results of the evaluation in Chapter~\ref{ch:eval}.
The last Chapter~\ref{ch:concl} concludes this work and gives an outlook for future research.

