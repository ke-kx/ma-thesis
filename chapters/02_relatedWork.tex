\chapter{Related Work}
%Position this here or just before conclusion? - right now here seems better though

How to split this in the best way?
-> check interesting related work and split it by topic / used technique / goal

General: read the summary paper \cite{robillard2013automated} again and check which approaches are there and which could / should be mentioned
+ check to read stuff\ldots

here: short intro to related work in general
Relevance of code smells and finding them!
first an overview of the different approaches that are commonly used for code smell detection
this includes static rules and bla
[summary of the other section]

\section{general approaches for bug detection / code smells}
What is the general state of the art (maybe extra section?)
static stuff, rules based, etc
mention findbugs?

\section{Section on approaches which learn from the given code}
Why is this potentially better than other approaches (actually learning an api vs static rules comes to mind)
also mention some stuff about recommender systems in general (it is the official thesis topic after all\ldots)

Mention \cite{engler2001bugs} as probably the first paper which proposed the general idea behind DMMC -> learning from the code at hand, instead of using static predefined rules

Remember the general machine learning approaches which were not super successful, but at least learned a LOT
especially like general code smells for example

\section{Previous work on detecting missing method calls / object usages}
-> api missuse detection

there have been a number of works concerned with finding patterns in object usages and using those to find potential bugs

short summaries of the two DMMC papers: \cite{monperrus2010detecting} and \cite{monperrus2013detecting}
most important for this works are the two papers by Monperrus et all (cite)
which introduced the notion of almost similarity and are the primary inspiration for this work
they consider the invocation of methods on an object of a particular type in a particular context a type usage
here the context is nothing more than the name of the method in which the object is used and its signature (type of its parameters)
after mining a list of all the type usages present in a code base, they relate the number of exactly equal type usages to the number of almost equal ones
exactly equal means that context, type and method list are identical and almost equal means the same only that the method list can contain one additional method
if there are a lot of almost equal type usages and very few equal ones, the tu under scrunity is probably an anomaly.
more details on their method can be found in the next section

results of monperrus et all:
% TODO do results!

Before this, Wasylkowski et al. \cite{wasylkowski2007detecting} introduced a method to locate anomalies in the order of methodcalls.
First, they extract usage models from Java code by building a finite state automata for each method.
The automata can be imagined similarly to the control flow graph of the method with instructions as transitions in the graph.
From these they mine temporal properties, which describe if a method $A$ can appear before another method $B$.
One can imagine this process as determining if there exists a path through the automata on which $A$ appears before $B$, which in turn implies an call to $A$ can happen before one to $B$.
Finally, they are using frequent itemset mining \cite{han2006data} to combine the temporal properties into patterns.
% todo add short explanation of frequent itemset mining? I think not really relevant

In this work an anomaly also occurs when many methods respect a pattern and only a few (a single one) break it.
In their experiments they find 790 violations when analyzing an open source program and find 790 violations.
Manual evaluation classifies those into 2 real defect, 5 smells and 84 ``hints'' (readability or maintainability could be improved).
This adds up to a fals positive rate of 87.8\%, but with an additional ranking method they were able to obtain the 2 defects and 3 out of 5 smells within the top 10 results.

In a related work Nguyen et al \cite{nguyen2009graph} use a graph-based representation of object usages to detect temporal dependencies.
This method stands out because it enables detecting dependencies between multiple objects and not just one.
% todo what does the branching thing even mean?
The object usages are represented as a labeled directed graph where the nodes are field accesses, constructor or method calls and branching is represented by control structures.
The edges of the graph represent the temporal usage order of methods and the dependencies between them.
% todo is this merge sort thing true?!
Patterns are then mined using a frequent induced subgraph detection algorithm which builds larger patterns from small patterns from the ground up, similar to the way merge sort operates.
Here an anomaly is also classifed as a ``rare'' violation of a pattern, i.e. it does not appear often in the dataset in relation to its size.
In an evaluation case study this work finds 64 defects in 9 open source software systems which the authors classifies to 5 true defects, 8 smells and 11 hints, which equals a false positive rate of 62.5\%.
Using a ranking method the top 10 results contain 3 defects, 2 smells and 1 hint.

% todo mention more related work from detecting mmc paper?

% \section{Section on Anomaly Detection}

\section{Android specific smell detection}

In the evaluation %todo ref!
we are analyzing a number of open source android applications.
As a small overview of android related code smell detection consider the following papers.

quick mention of the PAPRIKA paper mentioned in aDoctor intro?
    operates on byte code
    builds graph model from byte code and stores in db
    uses cypher query language to detect the smells -> basically hand coded rules as well
    only recognizes 4 android specific smells (and 4 general ones)
    --> understand the evalution / result section\ldots

Palomba et al. \cite{palomba2017lightweight}:
can detect 15 of 30 android specific smells proposed by Reimann et al. (cite), namely those which relate to a problem in the source code
works on abstract syntax tree of source code and uses specificly coded rules to detect each of the smells

evaluation: check 18 apps, compare analysis results of program against manually found smells
manual built oracle: 2 ppl read the code of the apps and flagged smells
average precision and recall 98\%

interesting results: the cases in which detection fails / yields false positives are exactly those where the hard coded detection rules don't consider a special case / a new library / \ldots

