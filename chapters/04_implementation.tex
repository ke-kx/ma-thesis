\chapter{Implementation}
details about implementation and pitfalls that had to be overcome

operates by statically analyzing a piece of software
(not on the source code, but rather on the compiled byte code -> why: easier)

\section{Bytecode Analysis}\label{sec:bytecode}
maybe rename section?

soot as analysis framework -> what is soot
why bytecode over sourcecode
\ldots

inaccuracy about ``object'' vs ``variable'' of a particular type?
-> there is the option for some analysis but it is super  expensive + seems leaky -> not doing it
but it will fix this (more or less)

\section{Procedure}

1. extract tus from software
2. for every tu:
    a) search for tus which are EXACTLY similar
    b) search for TUs which are almost similar
    c) compute the strangeness score
    d) extract list of potentially missing calls
3. output a list of tus, sorted by S score, the the ones with a score above XX are considered to be an anomaly + their missing calls

\section{Improvements}
using a database + flexible python benchmarking / analysis
-> proper overview of resulting system: java analysis -> db -> python
-> somewhere a proper overview of the concrete steps that are taken + explanation of them?! (maybe in extra chapter BEFORE this one?) - similar to monperrus2010 p7

explaining all the work i did
    first reading their code, coming across a couple of discrepancies between code and paper, later check if everthing still works (evaluation)
    refactoring everything + saving stuff to database
    building python infrastructure for analysis

\section{Dead Ends}
why some solutions where discarded (eg pure database / could be revisited if it turns out to be the best anyways - performance)
    clustering detector try
static functions evaluation!
something to fix the dotchaining problems

\section{Benchmark}

building benchmarking infrastructure
downloading + automatically analyzing android apps (in evaluation section?)

some changes that had to be made to the analysis framework for android analysis
