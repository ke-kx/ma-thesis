\chapter{Introduction}\label{chapter:introduction}

Some stuff about real life situation where missing method call can be a problem
-> find different bug report then the one the Majority Rule papers uses, but something like that

[Example]
bugs related to missing method calls can be found all over the internet
the issues they cause range from runtime exceptions to problems in limit cases
but generally reveal a code smell

\section{Motivation}<++>
% Motivating the research aka which problem are we trying to solve, general description of the problem

in an extensive analysis of the eclipse bug repository \cite{monperrus2013detecting} showed
that even in mature code there are many bugs related to missing method calls
developer not only spend time during development on bugs related to mm (Sandra example from paper)
bugs also survive development and get checked in to the repository

Thus it would be desirable to be able to detect missing method calls in production code
not only save expensive developer hours, also make maintenance cheaper and easier

% General idea why we want to use Recommender Systems / Learning!
A simple and straightforward approach would be to build a set of hard coded rules regarding method calls such as
``always call setControl() after instanciating a TextView''
``in Method ''onCreate`` of class extending ''XYZ`` always call ''foo.bar`)`` or
''when calling blub() also call blob()``
While well crafted and thought-out rules could allow for very high precision in detecting MMs, it would also require tremendous effort in maintaining the rule list
this kind of effort could be justified for big apis or libraries, but even then the ROI is questionable

To circumvent this problem, we would like to automatically and without needing any further input detect locations with potentially missing method calls
such an approach would adapt to a changing system without needing additional work and can also be applied to properitary code not open to the public
while the locations which will be found will probably not be 100\% accurate, they can then be examined by an expert who will determine the severity of the finding and issue a fix if necessary

-> much better than fixed preprogrammed rules, can adapt to changing system, be specific for own not open library, etc

the approach chosen in this work bases of recommender systems / learning
(Mention the ideas of \cite{bruch2012ide}, chapter 2 as an inspiration / the way to the idea - maybe)
first find likely recommendations, for writing, then realize, if something is super likely given a particular situation, but it is not there, it seems like a good indicator of an error

\section{Contribution}<++>

In this thesis we present a thorough reevaluation of the type usage characterization first introduced by Monperrus et al.\cite{monperrus2010detecting} and further refined in a follow up publication \cite{monperrus2013detecting}
type usage: list of method calls invoked on variable of a given type which occur in the body of a specific method
Majority rules idea: if a type is used in one particular way many many times and differently only one (or few) times, this probably indicates a bug

Further evaluation of the concept with comparison test using different similarity measures, application to big data set of open source android applications, \ldots

FINALLY: Summary of the other chapters of this thesis
