\chapter{Conclusion}\label{ch:concl}

% summary of what i did (maybe look at intro as well?)
In this thesis, we investigate a method for detecting missing method calls that was proposed by Monperrus et al.
Detecting missing method calls would be useful over the entirety of a software's lifetime, be it when developing a new program or when maintaining old and mature software.
The method we implement does not require any input besides the source code itself as it extracts patterns from the data it receives.
This makes it superior to other bug detection systems as it does not require the work of keeping hand-crafted detection rules up to date.

We improve upon the implementation of Monperrus et al.\ in that we add a database backend and enable better extensibility.
Furthermore, we propose several variations to their method with the goal of better detection of missing calls on the one hand and detecting different kind of errors on the other.
To get an all-round impression of the results that their method produces and how our variations behave in comparison, we evaluate them on a large dataset of Android applications. 

% general interpretation of ALL the results!
general aussage like ``its good or its useless''

	And of course relate the partition size graphs to the applicability on real world systems, like xku need a lot of data
	Can even be combined with the results about sensitivity to input pertubations
	Make it appear at least slightly clever

% experiments + results
\todo[inline]{
    Summary of results! + MAIN TAKEAWAY!
performed twofold evaluation, qualitatively and quantitatively
general takeaway about suitability of this method + reasons WHY I would say so
mention some numbers!

new method is amazing and way better in case xy but worse in zz
    tentively that it's inferior because of bad detection rate oä
robustness has shown this and that
}

\section{Future Work}

% general intro
\todo{smth with relation to how good i rate the method to be?}
In principle there are two dimensions along which we envision future research.
The first dimension concerns itself with improving the detection of missing method calls in some way.
The second dimension is about applying the majority rule to other types of anomalies.

% improve the method itself
As for improving the missing method call detection, we would first like to reference the analysis of typical failures we encountered that we did in Section ABC.
In it we already address a number of potential improvements as well as their drawbacks.
Especially some filters seem like an easy way to decrease the number of false positives and thus increase the precision, albeit they are a bit antithetical to the original idea of staying away from hand-crafted rule sets.
Another way to improve the precision could be to use a completely different anomaly detection algorithm.
\todo{examples?}
As mentioned, we did actually invest quite some time researching in that direction but it proved to be rather difficult, because the data on which to learn is actually rather thin.
\todo[inline]{rephrase\ldots -> didn't actually mention anywhere else!
mention in the end that i tried a 
Better anomaly detection (is the anomality rule actually a GOOD measure for this kind of anomalies, or should we use something totally different?))
    clustering detector try (+ hypersphere idea?)
    but all in all it didnt work out / not enough time
    + vermutungen warum es nicht funktioniert -> dataset exploration + size of method and typeusage lists per partition!}
\todo[inline]{correct reference!, mention examples? (dot chaining, static function etc)
make sure this paragraph works with the actual discussion section, stuff needs to be explained proper!\ldots}

We questioned how valid it is to partition the type usages by the context they appear in and suggested two other options for it ($\text{DMMC}_\text{noContext}$ and $\text{DMMC}_\text{class}$).
While our evaluation showed that grouping them by context produces the best results, there might be even better ways of organizing them.
We would be especially interested to see if one can improve the results by considering the inheritance hierarchy of the classes that the type usages appear in.

The majority rule is essentially looking for outliers by virtue of being different then the rest, the findings could certainly be improved by taking more data into consideration.
\todo{terrible first part of the sentence\ldots, actually terrible paragraph\ldots}
We already consider a sizeable dataset in our evaluation, however the ecosystem of existing Android applications is by far larger than that.
Analyzing a large portion of the Google playstore and extracting all of the type usages would certainly increase the precision.
We don't expect the performance to be a problem any time soon, however, as soon as any problems in that area emerge, it would be easy to optimize the implementation by using a dedicated database server and Cython (or any regular compiled language) for the analysis phase.

% apply to different anomalies
Besides these measures to improve the detection of missing method calls, one could also envision using the majority rules to detect other types of anomalies.
Monperrus et al.\ suggest applying concept of almost similarity to execution traces to detect runtime defects or to conditional statements to improve the resilience of software to incorrect input.
We also see potential in considering other features, for example, the interfaces a class implements or the methods it overrides.
\todo{idea clear?, any other pairings?, objectübergreifen?
right now siggy thought it better fits the first part - improving detection - which it kinda does
}

\todo[inline]{
    i think not mentioning this is actually smarter!
of course about temporal properties\ldots
Reihenfolge von Methoden - hat gut funktioniert und wäre potentiell nützlich, aber kleines subset + lots of research already done
order of method calls? (probably should wait for empirical results - how long are typical method lists (state explosion, ...) / latice solution?)
}
