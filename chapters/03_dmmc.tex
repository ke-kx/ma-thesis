\chapter{Detecting Missing Method Calls}

%--- quick intro again why it is interesting to detect them
thinking back to example in introduction, it is easy to see that there are bugs out there which occur because of missing method calls
additionally Monperrus et al. showed in their informal evaluation that real software systems have many bugs related to missing method calls

%--- overview of this chapter

\section{Majority Rule - existing work}
%--- start with intuition (restaurant example)
intutition behind the idea of the majority rule as follows
imagine being the waiter preparing the tables in a restaurant
there are 100 seats and each of them has a plate, a knife on the right (?) and a fork on the left
however only 99 of them have a spoon on the top,
one is missing the spoon.
then it is highly likely that this one exception to the rule: ``each seat should have a plate, a knife, a fork and a spoon'' is a mistake
and you should add the spoon to singular seat where it is missing

%--- formalized -> read the papers again for inspiration

%--- what is a type usage

%--- what are exactly equal and almost equal

%--- strangeness score calculation

%--- caveats
do real software systems actually behave in this way
ie: is it ``necessary'' to use classes in the same way
probably: most likely present in GUI systems

\section{Extensions / my work}
theoretical explanation of the different approaches
    load type usages in different ways
        per class type usages / ignore context
    look for different things (wrong method call, superfluous method)

    naive bayesian learner as baseline?
    something as of now unknown from anomaly detection
    clustering approach (hypersphere)
    working with the inheritance hierarchy!


