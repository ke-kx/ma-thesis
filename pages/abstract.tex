\chapter{\abstractname}

\todo[inline]{shorten! + mention software reuse?}
Software reuse is a central pillar of computer science (\ldots)
it enables large sytesm to rely on \ldots to do the core tasks while focussing on the specifics of their use case
-> way too long

Large software systems rely on frameworks and libraries to do the core work.
The application programming interface (API) they provide should be designed to be as easy to use as possible.
However, because of the great complexity and functionality provided by current frameworks, their APIs cannot be trivial.
This and the fact that their documentation is often not kept up to date leads to errors in invoking the APIs.

In the context of object-oriented programming, one of these errors related to API usages are missing method calls.
They occur when a developer correctly instantiates some object but forgets to call one or more of the necessary methods.
We study a new method for detecting missing method calls in Java applications which was proposed by Monperrus et al. \cite{monperrus2010detecting}.
The main goal is to understand how good this method is at detecting code smells and true bugs when applied to a large software system.
\todo{more concrete research question}

To investigate this, we analyze a large body of open source android applications.
Besides manual evaluation of the findings in X randomly selected apps, we also apply an automatic benchmark which relies on artificially degrading existing code.
\todo{describe results + concrete numbers}
% maybe not mention the degrading part?
We compare the originally proposed technique to some slight variations of it and find [\ldots].
The results are kind of mixed, but point in the direction of this method being marginally useful.
