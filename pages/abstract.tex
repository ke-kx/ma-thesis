\chapter{\abstractname}

Software reuse is a central pillar of software development, enabling developers to rely on frameworks and libraries to do the main work.
Because of the great complexity and functionality provided by current frameworks, often their application programming interface (API) cannot be trivial.
This and the fact that their documentation is often not kept up to date can led to errors when developers invoke these APIs. 

In the context of object-oriented programming, one of the errors related to API usages are missing method calls.
They occur when a developer correctly instantiates some object but forgets to call one or more of the necessary methods.
We study a new method for detecting missing method calls in Java applications which was proposed by Monperrus et al. \cite{monperrus2010detecting}.
The main goal is to understand how good this method is at detecting code smells and true bugs when applied to a large software system.
\todo{more concrete research question}

To investigate this, we analyze more than 600 open source android applications.
We manually evaluate the proposed findings of X randomly selected apps and also use an automatic benchmark which relies on artificially degrading existing code.
\todo{describe results + concrete numbers}
% maybe not mention the degrading part?
We compare the originally proposed technique to some slight variations of it and find [\ldots].
The results are kind of mixed, but point in the direction of this method being marginally useful.
