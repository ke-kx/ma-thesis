\chapter{Detecting Missing Method Calls}\label{ch:dmmc}

In the introduction, we explored errors related to missing method calls.
These occur frequently, particularly when developers are working with unfamiliar or new code.
\todo{or: Developers working with complicated frameworks and libraries or generally with resources they are new to are particularly prone to missing an important call.}
We would like to automatically detect instances of missed method calls to make maintenance easier and cheaper.
In this chapter, we explain in detail the method proposed by Monperrus et al.~\cite{monperrus2010detecting}\cite{monperrus2013detecting} which is based on the majority rule.

\section{Foundation: The Majority Rule}\label{sec:majority}

% general majority rule
The intuition behind the majority rule is that the way something is done most often is probably the correct way.
It follows that if everyone or almost everyone is doing something differently than you, you are probably doing it wrong.
\todo{(This is only relevant when the action being performed is in some way close to what you want to do.) -> smth more\ldots
    smth about how if it is too far away we don't care anymore at all (spoon / fork analogie\ldots)
}
This gives rise to the definition of an anomaly by the majority rule: instances with few or no ``equal'' but many ``slightly different'' neighbors.
Such anomalies should probably adapt to behave like their very similar neighbors.
What exactly ``equal'' and ``slightly different'' mean has to be defined for each application of the majority rule.

% example (comic sans + waiter)
To illustrate how this idea might work in the context of object-oriented software and method calls, consider the following example.
We are analyzing an Android app which uses a total of 100 instances of the \code{Button} class.
On all 100 the method \code{setText()} is invoked to describe the functionality of the button, but only 99 of them also call \code{setFont()}.
There is one button that is missing this call and, thus, uses the default font.
It seems highly likely that this one button is an exception to the rule ``each button which invokes \code{setText()} should also call \code{setFont()}'' and therefore a mistake.
To correct this, it should also make a call to \code{setFont()}.
Imagine all buttons in the application using a unique and beautiful font, but this one button displaying Comic Sans!

\section{Type Usages}

Generally speaking, type usages are an abstraction over the code.
They ignore the order of method calls and are only concerned with the list of method calls invoked on a particular object.
The critical pieces of information associated with a type usage are the type of the object, the list of methods invoked on it and the context in which the object is used.
We define the context as the name of the method in whose body the type usage appears together with the type of its parameters.
%todo I think the return type is ignored?

%--- formalized -> reread the papers for inspiration
It is useful here to define formal terms.
For each variable $x$, note its type $T(x)$ and the context $C(x)$ in which it occurs.
The list of methods which are invoked on $x$ within $C(x)$ is denoted as $M(x) =\{ m_1, m_2, \dotsc, m_n\}$.
If there are two variables of the same type, this results in two type usages being extracted (unless they refer to the same object; more on this in Section~\ref{sec:bytecode}).

\begin{figure}[h]
    \begin{subfigure}[c]{0.5\textwidth}
        \begin{lstlisting}[language=java, basicstyle=\small]
class A extends Page {
    Button b;

    Button createButton() {
        b = new Button();
        b.setText("hello");
        b.setColor(GREEN);
        ... (other code)
        Text t = new Text();
        return b;
    }
}
        \end{lstlisting}
    \end{subfigure}
    \begin{subfigure}[c]{0.5\textwidth}
        \begin{lstlisting}[basicstyle=\small]
T(b) = 'Button'
C(b) = 'Page.createButton()'
M(b) = {<init>, setText, setColor}

T(t) = 'Text'
C(t) = 'Page.createButton()'
M(t) = {<init>}
        \end{lstlisting}
    \end{subfigure}
    \caption{Example illustrating the Extraction of Type Usages (taken from \cite{monperrus2013detecting})}
    \label{fig:tu_example}
\end{figure}

As an example, consider the code in Figure~\ref{fig:tu_example}.
It contains two type usages, one of type \code{Button} and another of type \code{Text}.
The context for both is \code{createButton()}.
The methods invoked on the \code{Button} object are initialization (\code{new}), \code{setText}, and \code{setColor}.
The \code{Text} object, on the other hand, is only initialized.
Given the two variables $b$ and $t$ as input, the corresponding values of $T(x)$, $C(x)$ and $M(x)$ are shown on the right-hand side.

\section{Exact and Almost Similarity}

%--- what are exactly equal and almost equal
To detect type usages that are anomalous by the majority rule, we need a notion of what it means for two type usages to be exactly similar or only almost similar.
Informally, we say that two type usages are exactly similar if they have the same type, if the type usage appears in a similar method (i.e., the context is identical), and if their list of method calls is identical.
Recall that the context is the name of the method in which the type usage occurs, together with the type of its parameters.
We call these type usages ``similar'' instead of ``equal'' because several things are not the same: variable names, surrounding or interspersed code, method order or parameters, and the actual location (class).

The notion of almost similarity is analogous.
We say that a type usage $y$ is almost similar to a given type usage $x$ if they share the same type and if context and the list of method calls of $y$ is the same as the method calls of $x$ plus one additional call.

\begin{figure}[h]
    \centering
    \begin{subfigure}[c]{0.47\textwidth}
        \begin{lstlisting}[language=java, frame=single, basicstyle=\footnotesize]
class A extends Page {
    Button createButton() {
        Button b = new Button();
        ... (interlaced code)
        b.setText("hello");
        ... (interlaced code)
        b.setColor(BLUE);
        return b;
    }
}
        \end{lstlisting}
    \end{subfigure}
    \quad
    \begin{subfigure}[c]{0.47\textwidth}
        \begin{lstlisting}[language=java, frame=single, basicstyle=\footnotesize]
class B extends Page {
    Button createButton() {
        ... (code before)
        Button aBut = new Button();
        ... (interlaced code)
        aBut.setColor(RED);
        aBut.setText("goodbye");
        return b;
    }
}
        \end{lstlisting}
    \end{subfigure}
    \\

    \begin{subfigure}[c]{0.47\textwidth}
        \begin{lstlisting}[language=java, frame=single, basicstyle=\footnotesize, showstringspaces=false]
class C extends Page {
    Button myBut;
    Button createButton() {
        Button myBut = new Button();
        myBut.setColor(ORANGE);
        myBut.setText("Great Company!");
        myBut.setLink("https://www.cqse.eu");
        ... (code after)
        return b;
    }
}
        \end{lstlisting}
    \end{subfigure}
    \quad
    \begin{subfigure}[c]{0.47\textwidth}
        \begin{lstlisting}[language=java, frame=single, basicstyle=\footnotesize, showstringspaces=false]
class D extends Page {
    Button createButton() {
        Button tmp  = new Button();
        tmp.setLink("https://slashdot.org/");
        ... (interlaced code)
        tmp.setText("All the news");
        tmp.setColor(GREEN);
        tmp.setTooltipText("Click me!");
        return b;
    }
}
        \end{lstlisting}
    \end{subfigure}

    \caption{Examples of similar and almost similar Type Usages (also inspired by~\cite{monperrus2013detecting})}\label{fig:sim_asim}
\end{figure}

Consider the example code snippets in Figure~\ref{fig:sim_asim}.
The type usages of \code{Button} in classes \code{A} (top left) and \code{B} (top right) are exactly similar, because they not only appear in the similar method \code{createButton}, but also invoke the same methods \code{setText} and \code{setColor}.
Notice that variable names and method parameter have no impact on this.
Additionally, the comparison is not influenced by the order of methods or any interspersed code that does not call a method on the \code{Button} objects.
The type usage in class \code{C} (bottom left) is almost similar to those in \code{A} and \code{B}, because apart from \code{setText} and \code{setColor} it also invokes one additional method -- \code{setLink}.
Finally, the type usage in class \code{D} (bottom right) invokes \textbf{two} additional methods compared with the ones in \code{A} or \code{B}, meaning that it is not almost similar to them.
However, it is almost similar to the type usage in class \code{C}, since it only invokes one additional method -- \code{setTooltipText}.

%--- formalized
To formalize these notions, we define two binary relationships over type usages.
The relationship for exact similarity is called $E'$ and we say that two type usages $x$ and $y$ are exactly similar if and only if:
\begin{align*}
xE'y \iff & T(x) \: = T(y) \;\: \land \\
         & C(x) \: = C(y) \;\: \land \\
         & M(x) = M(y)
\end{align*}

Given this, the set of exactly similar type usages in relation to $x$ is defined as:
\begin{equation*}
E(x) = \{y \mid xE'y \}
\end{equation*}
Observe that this relation holds true for the identity, i.e., $xE'x$ is always true and $\forall x: x \in E(x)$ (and thus $|E(x)| \geq 1$).

For almost similarity, we define the relation $A'$ which holds if two type usages are almost similar.
A type usage $y$ is almost similar to a type usage $x$ iff:
\begin{align*}
xA'y \iff & T(x) \: = T(y) \;\; \land \\
         & C(x) \: = C(y) \;\; \land \\
         & M(x) \subset M(y) \: \land \\
         & |M(y)| = |M(x)| + 1
\end{align*}

Similarly to $E(x)$, we define $A(x)$ as the set of type usages which are almost similar to $x$:
\begin{equation*}
A(x) = \{y \mid xA'y \}
\end{equation*}

In contrast to $E(x)$, $A(x)$ can be empty and $|A(x)|\geq0$.
Consider also that $A'$ is not symmetrical, i.e., $xA'y \centernot\iff yA'x$ since one of the two type usages must have fewer methods.
This also means that $y \in A(x) \implies x \notin A(y)$.

We can further refine the definition of almost similarity.
Instead of including type usages which have the same method calls plus one additional one, it is also possible to consider those type usages which have the same method calls plus $k$ additional ones.
Then, in the definition of $A$ the last line changes to: $|M(y)| = |M(x)| + k, k\geq1$.
The purpose of this change would be to detect type usages which are missing $k$ calls instead of just one.

Given a codebase with $n$ type usages, the sets $E(x)$ and $A(x)$ for one given type usage $x$ can be computed in linear time $O(n)$ by iterating once through all type usages and checking for similarity or almost similarity.

\section{The Strangeness Score}

Recall the assumption behind the majority rule: A type usage is abnormal if a small number of type usages is exactly similar, but a significant number are almost similar.
Informally, this means a few places use this type in exactly the same way, but a significant majority use it in a way which is slightly different (by one method call).
Assuming the majority is correct, the type usage under scrutiny is deviant and potentially erroneous, because it is missing a method call.

To capture concretely how anomalous an object is, we need a measure of strangeness.
This will be the strangeness score.
We can use it to order type usages by how much of an outlier they are and to identify the ``strangest'' type usages, which are most interesting for evaluation by a human expert.

%--- strangeness score calculation
Formally, we define the S(trangeness)-Score as:
\begin{equation*}
    \operatorname{S-score}(x) = 1 - \frac{|E(x)|}{|E(x)|+|A(x)|}
\end{equation*}

To see that this definition makes sense, consider the following extreme cases:
Given one type usage $a$ without any additional exactly similar or almost similar type usages, it will have $|E(a)| = 1$ and $|A(a)| = 0$.
Then, we can calculate the strangeness score: $\operatorname{S-score}(a) = 1-\frac{1}{1} = 0$.
So a unique type usage without any ``neighbors'' to consider is completely normal and not strange.
On the other hand, given a type usage $b$ with $99$ almost similar and no other exactly similar type usages, we have $|E(b)| = 1$ and $|A(b)| = 99$.
This results in a strangeness score of $\operatorname{S-score}(b) = 1-\frac{1}{1+99} = 0.99$.
Intuitively such a type usage is very strange, and indeed, the S-score supports the intuition.

\section{Which Calls are missing?}

It is not enough to detect the type usages that are outliers.
We would also like to present the developer with some candidate suggestions, which method call might be missing.
The intuition for this is to look at all the almost similar type usages and collect the additional method calls that each of them invokes.
Recall, that an almost similar type usage will make exactly one additional call in comparison to the anomalous one.
We then take the frequency of unique calls among the set of additional method calls and use it as the suggestion likelihood.

Formally, the calls we can recommend for a type usage $x$ are the calls that are present in the type usages contained in $A(x)$ but are not in $M(x)$.
The set of these methods shall be called $R(x)$ and is defined as follows:
\begin{equation*}
    R(x) = \bigcup_{z \in A(x)} M(z) \setminus M(x)
\end{equation*}
% todo use this equation or the one in their paper (should be identical, which one is easier to follow?) 

%--- how to decide which is the call which is missing most likely
Now for each of these methods $m$ in $R(x)$, we can calculate their likelihood as follows:
\begin{equation*}
    \phi(m, x) = \frac{|\{z \mid z \in A(x) \land m \in M(z)\}|}{|A(x)|}
\end{equation*}

This is identical to the share of type usages which use this method among all the type usages which are almost similar to $x$.
Observe that this definition also works when $M(x)$ is empty or $x$ is \code{this}.

\begin{figure}[h]
    \begin{subfigure}[c]{0.45\textwidth}
        \begin{align*}
& T(x) = \mathtt{Button} \\
& M(x) = \{ \mathtt{<\!\!init\!\!>} \} \\
& A(x) = \{ a, b, c, d, e \} \\
& M(a) = \{ \mathtt{<\!\!init\!\!>},\, \mathtt{setText} \} \\
& M(b) = \{ \mathtt{<\!\!init\!\!>},\, \mathtt{setText} \} \\
& M(c) = \{ \mathtt{<\!\!init\!\!>},\, \mathtt{setText} \} \\
& M(d) = \{ \mathtt{<\!\!init\!\!>},\, \mathtt{setText} \} \\
& M(e) = \{ \mathtt{<\!\!init\!\!>},\, \mathtt{setFont} \}
        \end{align*}
    \end{subfigure}
    \begin{subfigure}[c]{0.45\textwidth}
        \begin{align*}
& R(x) = \{ \mathtt{setText},\, \mathtt{setFont} \} \\
& \phi(\mathtt{setText}, x) = \frac{4}{5} = 0.8 \\
& \phi(\mathtt{setFont}, x) = \frac{1}{5} = 0.2
        \end{align*}
    \end{subfigure}
    \caption{Example for computing the Likelihood of missing Calls (\cite{monperrus2013detecting})}\label{fig:missing_calls}
\end{figure}

To illustrate these definitions, consider the example in Figure~\ref{fig:missing_calls}.
The type usage under analysis is denoted with $x$, operates on a \code{Button} object and the only method that it invokes is the initialization.
There are five almost similar usages denoted as $a$, $b$, $c$, $d$ and $e$.
Four of them ($a$ through $d$) have a call to \code{setText} after the initialization.
The last one, $e$, has a call to \code{setFont} instead.
Thus, we can recommend the methods \code{setText} and \code{setFont} with a likelihood of $4/5 = 80\%$ and $1/5 = 20\%$ respectively.

\section{Ignoring the Context}

%--- why we use the context at all
We integrate the context into the definition of similarity and almost similarity because a type might be used in different ways depending on the situation it is used in.
An example would be the class \code{FileInputStream} which is rarely opened and closed in the same method.
In their evaluation, Monperrus et al.~\cite{monperrus2013detecting} find that not considering the context adds a lot of noise (not relevant items in $E(x)$ and $A(x)$) and using it improves the precision of their experiments.
%-- why it might be better to exclude the context
However, one can question the general assumption behind using the context.
The idea that types are used in different ways seems reasonable, but is the name of the function that they are used in a good separator between those use cases?
Furthermore, using the context might work well for some types and some systems, but does it make sense everywhere?
One problem with using the context is that it drastically reduces the number of type usages which are even considered for (almost) similarity.
In smaller datasets, this makes it much more likely that ``patterns'' appear, which in truth are only artifacts of randomness.

All in all, it is not clear that using the context necessarily improves the performance of the system.
Because of this, next to the standard variant $\mathnormal{DMMC}$ (Detecting Missing Method Calls), we also define a variant called $\mathnormal{DMMC}_{\mathnormal{noContext}}$, which does not take the context into account.
It uses adapted measures for similarity and almost similarity, $E'_{\mathnormal{noContext}}$ and $A'_{\mathnormal{noContext}}$ which are defined in the same manner as $E'$ and $A'$, except that they do not require the context of the type usages to be identical.
The definitions of $E$, $A$, $\operatorname{S-score}$ and $\phi$ stay the same, besides that they now make use of the newly defined relations.


